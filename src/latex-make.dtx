%    \iffalse meta-comment
%
%<*dtx>
          \ProvidesFile{latex-make.dtx}
[2021/10/26 v2.4.3 Python 3 as default]
%</dtx>
%    \fi
% \iffalse
%<*driver>
\documentclass[a4paper]{ltxdoc}
\usepackage{a4wide}
\usepackage[utf8]{inputenc}
\usepackage[T1]{fontenc}
\usepackage{pdfswitch}
\usepackage{lmodern}
\usepackage{array}
\usepackage{tabularx}
\usepackage{boxedminipage}
\EnableCrossrefs
\CodelineIndex
\RecordChanges
\setcounter{IndexColumns}{2}    % make a twocolumn index
\setlength{\columnseprule}{0pt} % no rules between columns ...
\setlength{\columnsep}{2em}     % ... but more spacing instead.
\setcounter{unbalance}{4}
\setlength{\IndexMin}{100pt}
%%%%\OnlyDescription  % uncomment this line to suppress printing the source code
\makeatletter
% don't show underfull hboxes in index/glossary:
\g@addto@macro{\IndexParms}{\hbadness=10000}
\g@addto@macro{\GlossaryParms}{\hbadness=10000}
\makeatother
\newenvironment{source}[1][.9\linewidth]{%
  \begin{center}%
    \begin{boxedminipage}[c]{#1}\tt%
      \vspace{1em}%
      \hspace{2em}\begin{minipage}[c]{#1}\tt%
      }{%
      \end{minipage}%
      \vspace{1em}%
    \end{boxedminipage}%
  \end{center}%
}
\sloppy
\begin{document}
  \DocInput{latex-make.dtx}
  \PrintIndex
  \PrintChanges
\end{document}
%</driver>
% \fi
%
% \GetFileInfo{latex-make.dtx}
%
% \DoNotIndex{\#,\$,\%,\&,\@,\\,\{,\},\^,\_,\~,\ }
% \DoNotIndex{\def,\long,\edef,\xdef,\gdef,\let,\global}
% \DoNotIndex{\if,\ifnum,\ifdim,\ifcat,\ifmmode,\ifvmode,\ifhmode,%
%             \iftrue,\iffalse,\ifvoid,\ifx,\ifeof,\ifcase,\else,\or,\fi,\loop,\do}
% \DoNotIndex{\box,\copy,\setbox,\unvbox,\unhbox,\hbox,%
%             \vbox,\vtop,\vcenter}
% \DoNotIndex{\@empty,\immediate,\write}
% \DoNotIndex{\egroup,\bgroup,\expandafter,\begingroup,\endgroup}
% \DoNotIndex{\divide,\advance,\multiply,\count,\dimen}
% \DoNotIndex{\relax,\space,\string}
% \DoNotIndex{\csname,\endcsname,\@spaces,\openin,\openout,%
%             \closein,\closeout}
% \DoNotIndex{\catcode,\endinput}
% \DoNotIndex{\jobname,\message,\read,\the,\noexpand}
% \DoNotIndex{\hsize,\vsize,\hskip,\vskip,\kern,\hfil,\hfill,\hss}
% \DoNotIndex{\m@ne,\z@,\@m,\z@skip,\@ne,\tw@,\p@}
% \DoNotIndex{\DeclareRobustCommand,\DeclareOption,\newcommand,\newcommand*}
% \DoNotIndex{\newcount,\newif,\newlinechar,\newread,\newtoks,\newwrite}
% \DoNotIndex{\dp,\wd,\ht,\vss,\unskip,\ifthenelse}
%
% \DoNotIndex{\@filef@und,\@nameddef,\@tempa}
% \DoNotIndex{\define@key,\equal,\ExecuteOptions}
% \DoNotIndex{\filename@area,\filename@base,\filename@ext,\filename@parse}
% \DoNotIndex{\Gin@exclamation,\Gin@getbase,\Gin@scalex,\Gin@scaley}
% \DoNotIndex{\Gread@eps,\Gread@pdf,\Gscale@box}
% \DoNotIndex{\IfFileExists,\ifpdf,\input,\InputIfFileExists}
% \DoNotIndex{\MessageBreak,\PackageWarning,\PackageWarningNoLine}
% \DoNotIndex{\ProcessOptions,\RequirePackage,\typeout}
% \DoNotIndex{\(,\),\.,\1,\t,\n,\^^J}
% \catcode\endlinechar 12\DoNotIndex{\
% }\catcode\endlinechar 10{}
% \catcode`\$=12
% \DoNotIndex{\$}
% \catcode`\$=3
% \DoNotIndex{\DeclareGraphicsExtensions,\DeclareGraphicsRule}
% 
%
% \title{The \LaTeX.mk Makefile\\
%   and related script tools\thanks{This file
%        has version number \fileversion, last
%        revised \filedate.}}
% \author{Vincent \textsc{Danjean} \and Arnaud \textsc{Legrand}}
% \date{\filedate}
% \maketitle
% \begin{abstract}
% This package allows to compile all kind and complex \LaTeX\space
% documents with the help of a Makefile. Dependencies are
% automatically tracked with the help of the |texdepends.sty| package.
% \end{abstract}
% \CheckSum{346}
%
% \changes{v2.0.0}{2006/03/09}{First autocommented version}
% \changes{v2.1.0}{2008/01/28}{That's the question}
% \changes{v2.1.1}{2009/11/08}{Improve error message}
% \changes{v2.1.2}{2012/03/17}{Switch from perl to python}
% \changes{v2.2.0}{2016/02/08}{Support to install LaTeX-Make locally}
% \changes{v2.2.1}{2016/02/09}{Improve configure}
% \changes{v2.2.2}{2016/02/09}{Fix bugs}
% \changes{v2.2.3}{2017/01/08}{Add LuaLaTeX support}
% \changes{v2.2.4}{2018/05/29}{Fix directory permissions on install}
% \changes{v2.2.5}{2018/09/04}{fix output format of figdepth.py}
% \changes{v2.3.0}{2018/10/17}{Add DEPENDS-EXCLUDE, add doc and
% support for local texmf tree}
% \changes{v2.4.0}{2020/06/01}{Support inkscape version >= 1.0}
% \changes{v2.4.1}{2020/07/10}{Fix encoding problem with latexfilter.pl}
% \changes{v2.4.2}{2021/01/03}{No changes in latex-make.dtx}
% \changes{v2.4.3}{2021/04/18}{Switch to T1 font encoding and lmodern font}
% \changes{v2.4.3}{2021/04/18}{Use python3 instead of python}
%
% \makeatletter
% \def\SpecialOptionIndex#1{\@bsphack
%     \index{#1\actualchar{\protect\ttfamily#1}
%            (option)\encapchar usage}%
%     \index{options:\levelchar{\protect\ttfamily#1}\encapchar
%            usage}\@esphack}
% \def\SpecialFileIndex#1{\@bsphack
%     \index{#1\actualchar{\protect\ttfamily#1}
%            (file)\encapchar usage}%
%     \index{files:\levelchar{\protect\ttfamily#1}\encapchar
%            usage}\@esphack}
% \def\SpecialMainOptionIndex#1{\@bsphack\special@index{#1\actualchar
%                                       {\string\ttfamily\space#1}
%                                          (option)\encapchar main}%
%     \special@index{options:\levelchar{%
%                    \string\ttfamily\space#1}\encapchar
%            main}\@esphack}
% \def\option{\begingroup
%    \catcode`\\12
%    \MakePrivateLetters \mym@cro@ \iffalse}
% \long\def\mym@cro@#1#2{\endgroup \topsep\MacroTopsep \trivlist
%    \edef\saved@macroname{\string#2}%
%   \def\makelabel##1{\llap{##1}}%
%   \if@inlabel
%     \let\@tempa\@empty \count@\macro@cnt
%     \loop \ifnum\count@>\z@
%       \edef\@tempa{\@tempa\hbox{\strut}}\advance\count@\m@ne \repeat
%     \edef\makelabel##1{\llap{\vtop to\baselineskip
%                                {\@tempa\hbox{##1}\vss}}}%
%     \advance \macro@cnt \@ne
%   \else  \macro@cnt\@ne  \fi
%   \edef\@tempa{\noexpand\item[%
%      #1%
%        \noexpand\PrintMacroName
%      \else
%        \noexpand\PrintEnvName
%      \fi
%      {\string#2}]}%
%   \@tempa
%   \global\advance\c@CodelineNo\@ne
%    #1%
%       \SpecialMainIndex{#2}\nobreak
%       \DoNotIndex{#2}%
%    \else
%       \SpecialMainOptionIndex{#2}\nobreak
%    \fi
%   \global\advance\c@CodelineNo\m@ne
%   \ignorespaces}
% \let\endoption \endtrivlist
% \def\DescribeOption{\leavevmode\@bsphack\begingroup\MakePrivateLetters
%   \Describe@Option}
% \def\Describe@Option#1{\endgroup
%               \marginpar{\raggedleft\PrintDescribeEnv{#1}}%
%               \SpecialOptionIndex{#1}\@esphack\ignorespaces}
% \def\DescribeFile{\leavevmode\@bsphack\begingroup\MakePrivateLetters
%   \Describe@Option}
% \def\Describe@File#1{\endgroup
%               \marginpar{\raggedleft\PrintDescribeEnv{#1}}%
%               \SpecialFileIndex{#1}\@esphack\ignorespaces}
% \makeatother
% \MakeShortVerb{\|}
%
% \tableofcontents
% \newpage
% \section{Introduction}
% 
% |latex-make| is a collection of \LaTeX{} packages, scripts and
% Makefile fragments that allows to easily compile \LaTeX{} documents.
% The best feature is that \textbf{\emph{dependencies are
%     automatically tracked}}\footnote{Dependencies are tracked with
%   the help of the |texdepend.sty| package that is automatically
%   loaded: no need to specify it with |\textbackslash usepackage\{\}|
%   in your documents.}.
% \par
% These tools can be used to compile small \LaTeX{} documents
% as well as big ones (such as, for example, a thesis with summary, tables of
% contents, list of figures, list of tabulars, multiple indexes and
% multiple bibliographies).
% 
% \section{Quick start}
% 
% \subsection{First (and often last) step}
% When you want to use |latex-make|, most of the time you have to
% create a |Makefile| with the only line: \par
% \begin{source}
%     include LaTeX.mk
% \end{source}
% % Then, the following targets are available: |dvi|, |ps|, |pdf|,
% \emph{file}|.dvi|, \emph{file}|.ps|, \emph{file}|.pdf|, etc.,
% |clean| and |distclean|.
% \par
%
% All \LaTeX{} documents of the current directory should be compilable
% with respect to their dependencies. If something fails, please,
% provide me the smallest example you can create to show me what is
% wrong.
% \par\medskip
%
% \paragraph{Tip:}
% If you change the dependencies inside your document (for example, if
% you change |\include{first}| into |\include{second}|), you may have
% to type |make distclean| before being able to recompile your
% document. Else, |make| can fail, trying to build or found the old
% |first.tex| file.
% \paragraph{Shared work}
% If you work with other people that do not have installed (and do not
% want to install) \LaTeX-Make, you can use the
% |LaTeX-Make-local-install| target in |LaTeX.mk| to install required
% files in a local TEXMF tree. You can them commit this
% tree into your control version system. Then, in your |Makefile|,
% replace the single line\par
% \begin{source}
%     include LaTeX.mk
% \end{source}
% with something like\par
% \begin{source}\small
%     export TEXMFHOME:=\$(CURDIR)/relpath/to/local/tree/texmf\\
%     include \$(shell env TEXMFHOME=\$(TEXMFHOME) $\backslash$\\
%     \hspace*{12ex}kpsewhich -format texmfscripts LaTeX.mk)
% \end{source}
% If you have a previous value for |TEXMFHOME| that you do not want to
% override, you can use the following (more complexe) snipset\par
% \begin{source}\small
%     \# Adapt the following line to find the local texmf tree\\
%     LOCAL\_TEXMF:=\$(CURDIR)/\$(firstword \$(wildcard texmf $\backslash$\\
%     \hspace*{16ex}../texmf ../../texmf ../../../texmf))\\
%     \# Get the old TEXMFHOME value\\
%     TEXMFHOME:=\$(shell kpsewhich -var-value TEXMFHOME)\\
%     \# If the new tree is already in it, do nothing, else add it\\
%     ifeq (\$(filter \$(LOCAL\_TEXMF)//%,\$(TEXMFHOME)),)\\
%     TEXMFHOME := \$(LOCAL\_TEXMF)\$(addprefix :,\$(TEXMFHOME))\\
%     \# display info so that users know what to define in order to\\
%     \# compile documents directly with (pdf)latex\\
%     \$(warning export TEXMFHOME=\$(TEXMFHOME))\\
%     export TEXMFHOME\\
%     endif\\
%     ~\\
%     include \$(shell env TEXMFHOME=\$(TEXMFHOME) $\backslash$\\
%     \hspace*{12ex}kpsewhich -format texmfscripts LaTeX.mk)
% \end{source}
% Doing so, all co-authors will be able to use \LaTeX-Make without
% installing it.  However, note that:
% \begin{itemize}
% \item you wont beneficit of an update of \LaTeX-Make in your system
%   (you will continue to use the locally installed files) ;
% \item there is no support for upgrading locally installed files (but
%   reexecuting the installation should do a correct upgrade most of
%   the time) ;
% \item  if a user tries to compile the \LaTeX{} source code directly
% with |[pdf]latex|, he must before either have LaTeX-Make installed
% or define and export |TEXMFHOME|.
% \end{itemize}
%
% Another possibility is to install package files (|*.sty|) into a
% directory pointed by |TEXINPUTS|, scripts files (|*.py|) into a
% directory pointed bt |TEXMFSCRIPTS|, and to directly include
% |LaTeX.mk|. For example:
% \begin{source}\small
%     \# use local files by default\\
%     \# packages in sty/ subdir and scripts in bin/\\
%     TEXINPUTS:=sty\$(addprefix :,\$(TEXINPUTS))::\\
%     TEXMFSCRIPTS:=bin\$(addprefix :,\$(TEXMFSCRIPTS))::\\
%     export TEXINPUTS\\
%     export TEXMFSCRIPTS\\
%     \# Force using local LaTeX.mk and not system-wide LaTeX.mk if available\\
%     include \$(CURDIR)/LaTeX.mk
% \end{source}
% \subsection{Customization}
% Of course, lots of things can be customized. Here are the most
% useful ones. Look at the section \ref{sec:reference} for more detailed
% and complete possibilities.
% \par
% Customization is done through variables in the |Makefile| set
% \emph{before} including |LaTeX.mk|. Setting them after can sometimes
% work, but not always and it is not supported.\par
%
% \newcommand{\variable}[4][\texttt]{%
%   \paragraph*{#3}%
%   \addcontentsline{toc}{subsubsection}{#3}%
%   \hfill \hbox{#1{\textbf{#2}}}\par\medskip%
%   \textbf{Example:}\hspace{1em} |#4|\par\medskip }
% 
% \variable{LU\_MASTERS}%
% {Which \LaTeX{} documents to compile}%
% {LU\_MASTERS=figlatex texdepends latex-make}%
% This variable contains the basename of the \LaTeX{} documents to
% compile. \par
% If not set, |LaTeX.mk| looks for all |*.tex| files containing the
% |\documentclass| command.
% 
% \variable[]{\emph{master}\texttt{\_MAIN}}%
% {Which \LaTeX{} main source for a document}%
% {figlatex\_MAIN=figlatex.dtx}%
% There is one such variable per documents declared in |LU_MASTERS|.
% It contains the file against which the |latex| (or |pdflatex|, etc.)
% program must be run. \par
% If not set, \emph{master}|.tex| is used.
% 
% \variable{LU\_FLAVORS}%
% {Which flavors must be compiled}%
% {LU\_FLAVORS=DVI DVIPDF}%
% A flavor can be see as a kind of document (postscript, PDF, DVI,
% etc.) and the way to create it. For example, a PDF document can be
% created directly from the |.tex| file (with |pdflatex|), from a
% |.dvi| file (with |dvipdfm|) or from a postscript file (with
% |ps2pdf|). This would be three different flavors.
% \par
% Some flavors are already defined in |LaTeX.mk|. Other flavors can be
% defined by the user (see section~\ref{sec:def_flavors}). The list of
% predefined flavors can be see in the table~\ref{tab:flavors}. A
% flavor can depend on another. For example, the flavor creating a
% postscript file from a DVI file depends on the flavor creating a DVI
% file from a \LaTeX{} file. This is automatically handled.
% \par
% If not set, |PS| and |PDF| are used (and |DVI| due to |PS|).
%\def\extrarowheight{2pt}
% \begin{table}[htbp]
%   \centering
%
%   \begin{tabular}{|c|c|l|l|}
%     \hline
%     Flavor & dependency & program variable & Transformation \\
%     \hline \hline
%     DVI & & LATEX & |.tex| $\Rightarrow$ |.dvi| \\
%     \hline
%     PS & DVI & DVIPS & |.dvi| $\Rightarrow$ |.ps| \\
%     \hline
%     PDF & & PDFLATEX & |.tex| $\Rightarrow$ |.pdf| \\
%     \hline
%     LUALATEX & & LUALATEX & |.tex| $\Rightarrow$ |.pdf| \\
%     \hline
%     DVIPDF & DVI & DVIPDFM & |.dvi| $\Rightarrow$ |.pdf| \\
%     \hline
%   \end{tabular}
%   \par\smallskip
%   \begin{minipage}[t]{0.8\linewidth}
%     \em For example, the |DVI| flavor transforms a |*.tex| file into
%     a |*.dvi| file with the |Makefile| command
%     |$(LATEX) $(LATEX_OPTIONS)|
%   \end{minipage}
%   %   \caption{Predefined flavors}
%   \label{tab:flavors}
% 
% \end{table}
%
% \variable[]{\emph{prog}/\emph{prog}\texttt{\_OPTIONS}}
% {Which programs are called and with which options}%
% {
%   \begin{minipage}[t]{0.5\linewidth}
%     DVIPS=dvips\\DVIPS\_OPTIONS=-t a4
%   \end{minipage}
% }
% Each flavor has a program variable name that is used by |LaTeX.mk|
% to run the program. Another variable with the suffix |\_OPTIONS| is
% also provided if needed. See the table~\ref{tab:flavors} the look
% for the program variable name associated to the predefined flavors.
% \par
% Other programs are also run in the same manner. For example, the
% |makeindex| program is run from |LaTeX.mk| with the help of the
% variables |MAKEINDEX| and |MAKEINDEX_OPTIONS|.
%
% \variable[]{\emph{master}\texttt{\_}\emph{prog}/\emph{master}\texttt{\_}\emph{prog}\texttt{\_OPTIONS}}
% {Per target programs and options}%
% {
%   \begin{minipage}[t]{0.5\linewidth}
%     figlatex\_DVIPS=dvips\\
%     figlatex\_DVIPS\_OPTIONS=-t a4
%   \end{minipage}
% }
% Note that, if defined, \emph{master}\texttt{\_}\emph{prog} will
% \textbf{\emph{replace}} \emph{prog} whereas
% \emph{master}\texttt{\_}\emph{prog}\texttt{\_OPTIONS} will
% \textbf{\emph{be added to}} \emph{prog}\texttt{\_OPTIONS} (see
% section \ref{sec:variables} for more details).
% 
% \variable[]{\texttt{DEPENDS}/\emph{master}\texttt{\_DEPENDS}}
% {Global and per target dependencies}%
% {
%   \begin{minipage}[t]{0.5\linewidth}
%     DEPENDS=texdepends.sty\\
%     figlatex\_DEPENDS=figlatex.tex
%   \end{minipage}
% }
% All flavor targets will depend to theses files. This should not be
% used as dependencies are automatically tracked.
%
%\section{Reference manual}
%\label{sec:reference}
% 
% \subsection{Flavors}
% \subsubsection{What is a flavor ?}
% A flavor can be see as a kind of document (postscript, PDF, DVI,
% etc.) and the way to create it. Several property are attached to
% each flavor. Currently, there exists two kinds of flavors:
% \begin{description}
% \item[TEX-flavors:] these flavors are used to compile a
% \texttt{*.tex} file into a target. A \LaTeX{} compiler
% (\texttt{latex}, \texttt{pdflatex}, etc.) is used;
% \item[DVI-flavors:] these flavors are used to compile a file
% produced by a TEX-flavors into an other file. Examples of such
% flavors are all the ones converting a DVI file into another format
% (postscript, PDF, etc.).
% \end{description}
% Several properties are attached to each flavors. Most are common, a
% few a specific to the kind of the flavor.
% \begin{description}
% \item[Name:] the name of the flavor. It is used to declare dependencies
% between flavors (see below). It also used to tell which flavor
% should be compiled for each document (see the \texttt{FLAVORS}
% variables);
% \item[Program variable name:] name of the variable that will be used
%   to run the program of this flavor. This name is used for the
%   program and also for the options (variable with the
%   \texttt{\_OPTIONS} suffix);
% \item[Target extension:] extension of the target of the flavor. The
% dot must be added if wanted;
% \item[Master target:] if not empty, all documents registered for the
% flavor will be built when this master target is called;
% \item[XFig extensions to clean (\emph{TEX-flavor only}):] files
%   extensions of figures that will be cleaned for the \texttt{clean}
%   target. Generally, there is \texttt{.pstex\_t .pstex} when using
%   \texttt{latex} and \texttt{.pdftex\_t .pdftex} when using
%   \texttt{pdflatex};
% \item[Dependency \emph{DVI-flavor only}:] name of the TEX-flavor the
%   one depends upon.
% \end{description}
%
% \subsubsection{Defining a new flavor}
% \label{sec:def_flavors}
% To define a new flavor named \texttt{NAME}, one just have to declare
% a \texttt{lu-define-flavor-NAME} that calls and evaluates the
% \texttt{lu-create-flavor} with the right parameters, ie:
% \begin{itemize}
% \item name of the flavor;
% \item kind of flavor (\texttt{tex} or \texttt{dvi});
% \item program variable name;
% \item target extension;
% \item master target;
% \item XFig extensions to clean \emph{or} TEX-flavor to depend upon.
% \end{itemize}
% 
% \par
%
% For example, \texttt{LaTeX.mk} already defines:
% \paragraph{DVI flavor}
% \begin{source}[0.9\linewidth]
% define lu-define-flavor-DVI\\
% \hspace*{2ex}\$\$(eval \$\$(call lu-create-flavor,DVI,tex,LATEX,.dvi,dvi,\textbackslash\\
% \hspace*{4ex}.pstex\_t .pstex))\\
% endef
% \end{source}
% \subparagraph{Tip:} the \texttt{LATEX} program variable name means
% that the program called will be the one in the \texttt{LATEX}
% variable and that options in the \texttt{LATEX\_OPTIONS} variable
% will be used.
%
% \paragraph{PDF flavor}
% \begin{source}[0.9\linewidth]
% define lu-define-flavor-PDF\\
% \hspace*{2ex}\$\$(eval \$\$(call lu-create-flavor,PDF,tex,PDFLATEX,.pdf,pdf,\textbackslash\\
% \hspace*{4ex}.pdftex\_t .\$\$(\_LU\_PDFTEX\_EXT)))\\
% endef
% \end{source}
% 
% \paragraph{LuaLaTeX flavor}
% \begin{source}[0.9\linewidth]
% define lu-define-flavor-LUALATEX\\
% \hspace*{2ex}\$\$(eval \$\$(call lu-create-flavor,LUALATEX,tex,LUALATEX,.pdf,pdf,\textbackslash\\
% \hspace*{4ex}.pdftex\_t .\$\$(\_LU\_PDFTEX\_EXT)))\\
% endef
% \end{source}
% 
% \paragraph{PS flavor}
% \begin{source}[0.9\linewidth]
% define lu-define-flavor-PS\\
% \hspace*{2ex}\$\$(eval \$\$(call lu-create-flavor,PS,dvi,DVIPS,.ps,ps,DVI))\\
% endef
% \end{source}
% \subparagraph{Tip:} for DVI-flavors, the program will be invoked with
% with the option \texttt{-o \emph{target}} and with the name of the
% file source in argument.
% 
% \paragraph{DVIPDF flavor}
% \begin{source}[0.9\linewidth]
% define lu-define-flavor-DVIPDF\\
% \hspace*{2ex}\$\$(eval \$\$(call lu-create-flavor,DVIPDF,dvi,DVIPDFM,.pdf,pdf,DVI))\\
% endef
% \end{source}
% 
%
% \subsection{Variables}
% \label{sec:variables}
% \texttt{LaTeX.mk} use a generic mechanism to manage variables, so
% that lots of thing can easily be customized per document and/or per
% flavor.
% \subsubsection{Two kind of variables}
% \texttt{LaTeX.mk} distinguish two kind of variables. The first one
% (called SET-variable) is for variables where only \emph{one} value
% can be set. For example, this is the case for a variable that
% contain the name of a program to launch. The second one (called
% ADD-variable) is for variables where values can be cumulative. For
% example, this will be the case for the options of a program.
% \par
%
% For each variable used by \texttt{LaTeX.mk}, there exists several
% variables that can be set in the Makefile so that the value will be
% used for all documents, only for one document, only for one flavor,
% etc.
%
% \newcounter{nbvars}%
% \newenvironment{descvars}{%
%   \par
%   \noindent%
%   \bgroup%
%   \setcounter{nbvars}{0}%
%   \renewcommand{\emph}[1]{{\it ##1}}
%   \renewcommand{\variable}[3][;]{%
%     \stepcounter{nbvars}%
%     \arabic{nbvars}&\textbf{##2}&##3##1\\%
%   }
%   \begin{tabular}{clp{0.5\linewidth}}
%   }{%
%   \end{tabular}
%   \egroup
%   \par\medskip
% }
%
% \paragraph{SET-variable.} For each SET-variable \texttt{\emph{NAME}}, we
% can find in the Makfile:
% \begin{descvars}
%   \variable{\texttt{LU\_\emph{target}\_\emph{NAME}}}{per document
%     and per flavor value}%
%   \variable{\texttt{TD\_\emph{target}\_\emph{NAME}}}{per document
%     and per flavor value filled by the \texttt{texdepends} \LaTeX{}
%     package}%
%   \variable{\texttt{LU\_\emph{master}\_\emph{NAME}}}{per document
%     value}%
%   \variable{\texttt{\emph{master}\_\emph{NAME}}}{per document
%     value}%
%   \variable{\texttt{LU\_FLAVOR\_\emph{flavor}\_\emph{NAME}}}{per
%     flavor value}%
%   \variable{\texttt{LU\_\emph{NAME}}}{global value}%
%   \variable{\texttt{\emph{NAME}}}{global value}%
%   \variable[.]{\texttt{\_LU\_\ldots\emph{NAME}}}{internal
%     \texttt{LaTeX.mk} default values}%
% \end{descvars}
% The first set variable will be used.
%
% \subparagraph{Tip:} in case of flavor context or document context,
% only relevant variables will be checked. For example, the
% SET-variable \texttt{MAIN} that give the main source of the document
% will be evaluated in document context, so only 4, 5, 6, 7 and 8 will
% be used (and I cannot see any real interest in using 6 or 7 for this
% variable).
% 
% \subparagraph{Tip2:} in case of context of index (when building
% indexes or glossary), there exists several other variables per index
% to add to this list (mainly ending with \texttt{\_{\it kind}\_{\it
%     indexname}\_{\it NAME}} or \texttt{\_{\it kind}\_{\it NAME}}).
% Refer to the sources if you really need them.
%
% \paragraph{ADD-variable.} An ADD-variable is cumulative. The user
% can replace or add any values per document, per flavor, etc.
% \begin{descvars}
%   \variable{\texttt{LU\_\emph{target}\_\emph{NAME}}}{replacing per
%     document and per flavor values}%
%   \variable{\texttt{\emph{target}\_\emph{NAME}}}{cumulative per
%     document and per flavor values}%
%   \variable{\texttt{LU\_\emph{master}\_\emph{NAME}}}{replacing per
%     document values}%
%   \variable{\texttt{\emph{master}\_\emph{NAME}}}{cumulative per
%     document values}%
%   \variable{\texttt{LU\_FLAVOR\_\emph{flavor}\_\emph{NAME}}}{replacing
%     per flavor values}%
%   \variable{\texttt{FLAVOR\_\emph{flavor}\_\emph{NAME}}}{cumulative
%     per flavor values}%
%   \variable{\texttt{LU\_\emph{NAME}}}{replacing global values}%
%   \variable{\texttt{\emph{NAME}}}{cumulative global values}%
% \end{descvars}
% \subparagraph{Tip:} if not defined, \texttt{LU\_\emph{variable}}
% defaults to ``\texttt{\$(\emph{variable})
%   \$(\_LU\_\emph{variable})}'' and \texttt{\_LU\_\emph{variable}}
% contains default values managed by \texttt{LaTeX.mk} and the
% \texttt{texdepends} \LaTeX{} package.
% \subparagraph{Example:} the ADD-variable \texttt{FLAVORS} is invoked
% in document context to know which flavors needs to be build for each
% document. This means that \texttt{LU\_{\it master}\_FLAVORS} will be
% used.
% \begin{source}
%   \# We override default value for MASTERS\\
%   LU\_MASTERS=foo bar baz\\
%   \# By default, only the DVIPDF flavor will be build\\
%   FLAVORS=DVIPDF\\
%   ~\\
%   bar\_FLAVORS=PS\\
%   LU\_baz\_FLAVORS=PDF\\
%   \# there will be rules to build\\
%   \# * foo.dvi and foo.pdf\\
%   \# ~ (the DVIPDF flavor depends on the DVI flavor)\\
%   \# * bar.dvi, bar.pdf and bar.ps\\
%   \# ~ (the PS flavor is added to global flavors)\\
%   \# * baz.pdf\\
%   \# ~ (the PDF flavor will be the only one for baz)
%   include LaTeX.mk
% \end{source}
% \subsubsection{List of used variables}
% Here are most of the variables used by \texttt{LaTeX.mk}. Users
% should only have to sometimes managed the first ones. The latter are
% described here for information only (and are subject to
% modifications). Please, report a bug if some of them are not
% correctly pickup by the \texttt{texdepends} \LaTeX{} package and
% \texttt{LaTeX.mk}.
%
% \newenvironment{describevars}{%
%   \par
%   \noindent%
%   \bgroup%
%   \renewcommand{\emph}[1]{{\it ##1}}%
%   \newcommand{\default}[1]{\newline\textbf{Default: }##1}
%   \renewcommand{\variable}[4]{%
%     \textbf{\texttt{##1}}&##2&##3&##4\\%
%   }
%   \begin{tabular}{cccp{0.5\linewidth}}
%     Name & Kind &%
%     \multicolumn{1}{m{4em}}{\begin{center}Context of
%         use\end{center}} &
%     \multicolumn{1}{c}{Description} \\
%     \hline
%     \hline
%   }{%
%   \end{tabular}
%   \egroup
%   \par\medskip
% }
%
% \begin{describevars}
%   \variable{MASTERS}{ADD}{Global}{List of documents to compile.
%     These values will be used as jobname. \default{basename of
%       \texttt{*.tex} files containing the \texttt{\textbackslash
%         documentclass} pattern}}%
%   \variable{FLAVORS}{ADD}{Document}{List of flavors for each
%     document.  \default{\texttt{PS PDF}}}%
%   \variable{MAIN}{SET}{Document}{Master \texttt{tex} source
%     file\default{\texttt{\emph{master}.tex}}}%
%   \variable{DEPENDS}{ADD}{Target}{List of dependencies}%
%   \variable{DEPENDS\_EXCLUDE}{ADD}{Target}{Dependencies to
%   forget. Useful when LaTeX Make wrongly auto-detect false dependencies}%
%   \variable{\emph{progvarname}}{SET}{Target}{Program to launch for
%     the corresponding flavor}%
%   \variable{\emph{progvarname}\_OPTIONS}{ADD}{Target}{Options to use
%     when building the target}%
%   \variable{STYLE}{SET}{Index}{Name of the index/glossary style file
%     to use (\texttt{.ist}, etc.)}%
%   \hline
%   \variable{TARGET}{SET}{Index}{Name of the index/glossary file to
%     produce (\texttt{.ind}, \texttt{.gls}, etc.)}%
%   \variable{SRC}{SET}{Index}{Name of the index/glossary file source
%     (\texttt{.idx}, \texttt{.glo}, etc.)}%
%   \variable{FIGURES}{ADD}{Target}{Lists of figures included}%
%   \variable{BIBFILES}{ADD}{Target}{Lists of bibliography files used
%     (\texttt{.bib})}%
%   \variable{BIBSTYLES}{ADD}{Target}{Lists of bibliography style
%     files used (\texttt{.bst})}%
%   \variable{BBLFILES}{ADD}{Target}{Lists of built bibliography files
%     (\texttt{.bbl})}%
%   \variable{INPUT}{ADD}{Target}{Lists of input files (\texttt{.cls},
%     \texttt{.sty}, \texttt{.tex}, etc.)}%
%   \variable{OUTPUTS}{ADD}{Target}{Lists of output files
%     (\texttt{.aux}, etc.)}%
%   \variable{GRAPHICSPATH}{ADD}{Target}{\texttt{\textbackslash
%       graphicspath\{\}} arguments}%
%   \variable{GPATH}{ADD}{Target}{List of directories from
%     \texttt{GRAPHICSPATH} without \texttt{\{} and \texttt{\}},
%     separated by spaces}%
%   \variable{INDEXES}{ADD}{Target}{Kinds of index (\texttt{INDEX},
%     \texttt{GLOSS}, etc.)}%
%   \variable{INDEXES\_\emph{kind}}{ADD}{Target}{List of indexes or
%     glossaries}%
%   \variable{WATCHFILES}{ADD}{Target}{List of files that trigger a
%     rebuild if modified (\texttt{.aux}, etc.)}%
%   \variable{REQUIRED}{ADD}{Target}{List of new dependencies found by
%     the \texttt{texdepends} \LaTeX{} package}%
%   \variable{MAX\_REC}{SET}{Target}{Maximum level of recursion
%     authorized}%
%   \variable{REBUILD\_RULES}{ADD}{Target}{List of rebuild rules to
%     use (can be modified by the \texttt{texdepends} \LaTeX{}
%     package}%
%   \variable{EXT}{SET}{Flavor}{Target file extension of the flavor}%
%   \variable{DEPFLAVOR}{SET}{Flavor}{TEX-flavor a DVI-flavor depend
%     upon}%
%   \variable{CLEANFIGEXT}{ADD}{Flavor}{Extensions of figure files to
%     remove on clean}%
% \end{describevars}
%
% \newpage
% \section{FAQ}
%
% \newenvironment{question}[2]{
%   \subsection{#1}
%   \texttt{}\llap{$\Rightarrow$ \quad }\emph{#2}\par\bigskip}{
% }
%
% \begin{question}{No rule to make target `LU\_WATCH\_FILES\_SAVE'}{When
%   using |LaTeX.mk|, I got the error:\\
%   |make[1]: *** No rule to make target `LU\_WATCH\_FILES\_SAVE'. Stop.|}
%
% |make| is called in such a way that does not allow correct recursive
% calls. As one can not know by advance how many times \LaTeX{},
% bib\TeX{}, etc. will need to be run, |latex-make| use recursive
% invocations of |make|. This means that in the |LaTeX.mk| makefile,
% there exist rules such as:
% \begin{source}
%   \$(MAKE) INTERNAL\_VARIABLE=value internal\_target
% \end{source}
% In order |latex-make| to work, this invocation of |make| must read
% the same rules and variable definitions as the main one. This means
% that calling "|make -f LaTeX.mk foo.pdf|" in a directory with only
% |foo.tex| will not work. Recursive invocations of make will not load
% |LaTeX.mk|, will search for a |Makefile| in the current directory
% and will complain about being unable to build the
% |LU_WATCH_FILES_SAVE| internal target.
% 
% \par\medskip
% The solution is to call |make| so that recursive invocations will
% read the same variables and rules. For example:\\
% |make -f LaTeX.mk MAKE="make -f LaTeX.mk" foo.pdf|\\
% or (if there is no |Makefile| in the directory):\\
% |env MAKEFILES=LaTeX.mk make foo.pdf|\\
%
% \end{question}
%
% \StopEventually{
% }
% \newpage
% \section{Implementation}
%
% \subsection{LaTeX.mk}
%    \begin{macrocode}
%<*makefile>

####[ Check Software ]################################################

ifeq ($(filter else-if,$(.FEATURES)),)
$(error GNU Make 3.81 needed. Please, update your software.)
	exit 1
endif

# Some people want to call our Makefile snippet with
# make -f LaTeX.mk
# This should not work as $(MAKE) is call recursively and will not read
# LaTeX.mk again. We cannot just add LaTeX.mk to MAKEFILES as LaTeX.mk
# should be read AFTER a standard Makefile (if any) that can define some
# variables (LU_MASTERS, ...) that LaTeX.mk must see.
# So I introduce an HACK here that try to workaround the situation. Keep in
# mind that this hack is not perfect and does not handle all cases
# (for example, "make -f my_latex_config.mk -f LaTeX.mk" will not recurse
# correctly)
ifeq ($(foreach m,$(MAKEFILES), $(m)) $(lastword $(MAKEFILE_LIST)),$(MAKEFILE_LIST))
# We are the first file read after the ones from MAKEFILES
# So we assume we are read due to "-f LaTeX.mk"
LU_LaTeX.mk_NAME := $(lastword $(MAKEFILE_LIST))
# Is this Makefile correctly read for recursive calls ?
ifeq ($(findstring -f $(LU_LaTeX.mk_NAME),$(MAKE)),)
$(info ********************************************************************************)
$(info Warning: $(LU_LaTeX.mk_NAME) called directly. I suppose that you run:)
$(info Warning: $(MAKE) -f $(LU_LaTeX.mk_NAME) $(MAKECMDGOALS))
$(info Warning: or something similar that does not allow recursive invocation of make)
$(info Warning: )
$(info Warning: Trying to enable a workaround. This ACK will be disabled in a future)
$(info Warning: release. Consider using another syntax, for example:)
$(info Warning: $(MAKE) -f $(LU_LaTeX.mk_NAME) MAKE="$(MAKE) -f $(LU_LaTeX.mk_NAME)" $(MAKECMDGOALS))
$(info ********************************************************************************)
MAKE+= -f $(LU_LaTeX.mk_NAME)
endif
endif

####[ Configuration ]################################################

# list of messages categories to display
LU_SHOW ?= warning #info debug debug-vars

# Select GNU/BSD/MACOSX utils (cp, rm, mv, ...)
LU_UTILS ?= $(shell ( /bin/cp --heelp > /dev/null 2>&1 && echo GNU ) || echo BSD )
export LU_UTILS

####[ End of configuration ]################################################
# Modifying the remaining of this document may endanger you life!!! ;)

#---------------------------------------------------------------------
# Controling verbosity
ifdef VERB
MAK_VERB :=  $(VERB)
else
#MAK_VERB :=  debug
#MAK_VERB :=  verbose
#MAK_VERB :=  normal
MAK_VERB :=  quiet
#MAK_VERB :=  silent
endif

#---------------------------------------------------------------------
# MAK_VERB -> verbosity
ifeq ($(MAK_VERB),debug)
COMMON_PREFIX  =  echo "         ======> building " $@ "<======" ; \
	printf "%s $(@F) due to:$(foreach file,$?,\n      * $(file))\n" $1; set -x;
#
COMMON_HIDE   := set -x;
COMMON_CLEAN  := set -x;
SHOW_LATEX:=true
else
ifeq ($(MAK_VERB),verbose)
COMMON_PREFIX  =  echo "         ======> building " $@ "<======" ; \
	printf "%s $(@F) due to:$(foreach file,$?,\n      * $(file))\n" $1; 
#
COMMON_HIDE   :=#
COMMON_CLEAN  :=#
SHOW_LATEX:=true
else
ifeq ($(MAK_VERB),normal)
COMMON_PREFIX  =#
COMMON_HIDE   :=  @
COMMON_CLEAN  :=#
SHOW_LATEX:=true
else
ifeq ($(MAK_VERB),quiet)
COMMON_PREFIX  =  @ echo "         ======> building " $@ "<======" ;
#		echo "due to $?" ;
COMMON_HIDE   :=  @
COMMON_CLEAN  :=#
SHOW_LATEX:=
else  # silent
COMMON_PREFIX  =  @
COMMON_HIDE   :=  @
COMMON_CLEAN  :=  @
SHOW_LATEX:=
endif
endif
endif
endif

#---------------------------------------------------------------------
# Old LaTeX have limitations
_LU_PDFTEX_EXT ?= pdftex

#########################################################################
# Utilities
LU_CP=$(LU_CP_$(LU_UTILS))
LU_MV=$(LU_MV_$(LU_UTILS))
LU_RM=$(LU_RM_$(LU_UTILS))
LU_CP_GNU ?= cp -a --
LU_MV_GNU ?= mv --
LU_RM_GNU ?= rm -f --
LU_CP_BSD ?= cp -p
LU_MV_BSD ?= mv
LU_RM_BSD ?= rm -f
LU_CP_MACOSX ?= /bin/cp -p
LU_MV_MACOSX ?= /bin/mv
LU_RM_MACOSX ?= /bin/rm -f

lu-show=\
$(if $(filter $(LU_SHOW),$(1)), \
	$(if $(2), \
		$(if $(filter-out $(2),$(MAKELEVEL)),,$(3)), \
		$(3)))
lu-show-infos=\
$(if $(filter $(LU_SHOW),$(1)), \
	$(if $(2), \
		$(if $(filter-out $(2),$(MAKELEVEL)),,$(warning $(3))), \
		$(warning $(3))))
lu-show-rules=$(call lu-show-infos,info,0,$(1))
lu-show-flavors=$(call lu-show-infos,info,0,$(1))
lu-show-var=$(call lu-show-infos,debug-vars,,  * Set $(1)=$($(1)))
lu-show-read-var=$(eval $(call lu-show-infos,debug-vars,,  Reading $(1) in $(2) ctx: $(3)))$(3)
lu-show-readone-var=$(eval $(call lu-show-infos,debug-vars,,  Reading $(1) for $(2) [one value]: $(3)))$(3)
lu-show-set-var=$(call lu-show-infos,debug-vars,,  * Setting $(1) for $(2) to value: $(3))
lu-show-add-var=$(call lu-show-infos,debug-vars,,  * Adding to $(1) for $(2) values: $(value 3))
lu-show-add-var2=$(call lu-show-infos,warning,,  * Adding to $(1) for $(2) values: $(value 3))

lu-save-file=$(call lu-show,debug,,echo "saving $1" ;) \
	if [ -f "$1" ];then $(LU_CP) "$1" "$2" ;else $(LU_RM) "$2" ;fi
lu-cmprestaure-file=\
	if cmp -s "$1" "$2"; then \
		$(LU_MV) "$2" "$1" ; \
		$(call lu-show,debug,,echo "$1" not modified ;) \
	else \
		$(call lu-show,debug,,echo "$1" modified ;) \
		if [ -f "$2" -o -f "$1" ]; then \
			$(RM) -- "$2" ; \
			$3 \
		fi ; \
	fi

lu-clean=$(if $(strip $(1)),$(RM) $(1))

define lu-bug # description
  $$(warning Internal error: $(1))
  $$(error You probably found a bug. Please, report it.)
endef

#########################################################################
#########################################################################
#########################################################################
#########################################################################
##################                              #########################
##################          Variables           #########################
##################                              #########################
#########################################################################
#########################################################################
#########################################################################
#########################################################################
#########################################################################
#
# _LU_FLAVORS_DEFINED : list of available flavors
# _LU_FLAV_*_'flavname' : per flavor variables
#   where * can be :
#   PROGNAME : variable name for programme (and .._OPTIONS for options)
#   EXT : extension of created file
#   TARGETNAME : global target
#   DEPFLAVOR : flavor to depend upon
#   CLEANFIGEXT : extensions to clean for fig figures      
_LU_FLAVORS_DEFINED = $(_LU_FLAVORS_DEFINED_TEX) $(_LU_FLAVORS_DEFINED_DVI)

# INDEXES_TYPES = GLOSS INDEX
# INDEXES_INDEX = name1 ...
# INDEXES_GLOSS = name2 ...
# INDEX_name1_SRC
# GLOSS_name2_SRC

define _lu-getvalues# 1:VAR 2:CTX (no inheritage)
$(if $(filter-out undefined,$(origin LU_$2$1)),$(LU_$2$1),$($2$1) $(_LU_$2$1_MK) $(TD_$2$1))
endef
define lu-define-addvar # 1:suffix_fnname 2:CTX 3:disp-debug 4:nb_args 5:inherited_ctx 6:ctx-build-depend
  define lu-addtovar$1 # 1:VAR 2:... $4: value
    _LU_$2$$1_MK+=$$($4)
    $$(call lu-show-add-var,$$1,$3,$$(value $4))
  endef
  define lu-def-addvar-inherited-ctx$1 # 1:VAR 2:...
    $6
    _LU_$2$$1_INHERITED_CTX=$$(sort \
      $$(foreach ctx,$5,$$(ctx) $$(if $$(filter-out undefined,$$(origin \
          LU_$$(ctx)$$1)),,\
         $$(_LU_$$(ctx)$$1_INHERITED_CTX))))
    $$$$(call lu-show-var,_LU_$2$$1_INHERITED_CTX)
  endef
  define lu-getvalues$1# 1:VAR 2:...
$$(if $$(filter-out undefined,$$(origin _LU_$2$$1_INHERITED_CTX)),,$$(eval \
  $$(call lu-def-addvar-inherited-ctx$1,$$1,$$2,$$3,$$4,$$5,$$6)\
))$$(call lu-show-read-var,$$1,$3,$$(foreach ctx,\
    $(if $2,$2,GLOBAL) $$(if $$(filter-out undefined,$$(origin LU_$2$$1)),,\
             $$(_LU_$2$$1_INHERITED_CTX))\
    ,$$(call _lu-getvalues,$$1,$$(filter-out GLOBAL,$$(ctx)))))
  endef
endef

# Global variable
# VAR (DEPENDS)
$(eval $(call lu-define-addvar,-global,,global,2))

# Per flavor variable
# FLAVOR_$2_VAR (FLAVOR_DVI_DEPENDS)
# 2: flavor name
# Inherit from VAR (DEPENDS)
$(eval $(call lu-define-addvar,-flavor,FLAVOR_$$2_,flavor $$2,3,\
  GLOBAL,\
  $$(eval $$(call lu-def-addvar-inherited-ctx-global,$$1)) \
))

# Per master variable
# $2_VAR (source_DEPENDS)
# 2: master name
# Inherit from VAR (DEPENDS)
$(eval $(call lu-define-addvar,-master,$$2_,master $$2,3,\
  GLOBAL,\
  $$(eval $$(call lu-def-addvar-inherited-ctx-global,$$1)) \
))

# Per target variable
# $2$(EXT of $3)_VAR (source.dvi_DEPENDS)
# 2: master name
# 3: flavor name
# Inherit from $2_VAR FLAVOR_$3_VAR (source_DEPENDS FLAVOR_DVI_DEPENDS)
$(eval $(call lu-define-addvar,,$$2$$(call lu-getvalue-flavor,EXT,$$3)_,target $$2$$(call lu-getvalue-flavor,EXT,$$3),4,\
  $$2_ FLAVOR_$$3_,\
  $$(eval $$(call lu-def-addvar-inherited-ctx-master,$$1,$$2)) \
  $$(eval $$(call lu-def-addvar-inherited-ctx-flavor,$$1,$$3)) \
))

# Per index/glossary variable
# $(2)_$(3)_VAR (INDEX_source_DEPENDS)
# 2: type (INDEX, GLOSS, ...)
# 3: index name
# Inherit from VAR (DEPENDS)
$(eval $(call lu-define-addvar,-global-index,$$2_$$3_,index $$3[$$2],4,\
  GLOBAL,\
  $$(eval $$(call lu-def-addvar-inherited-ctx-global,$$1)) \
))

# Per master and per index/glossary variable
# $(2)_$(3)_$(4)_VAR (source_INDEX_source_DEPENDS)
# 2: master name
# 3: type (INDEX, GLOSS, ...)
# 4: index name
# Inherit from $2_VAR $3_$4_VAR (source_DEPENDS INDEX_source_DEPENDS)
$(eval $(call lu-define-addvar,-master-index,$$2_$$3_$$4_,index $$2/$$4[$$3],5,\
  $$2_ $$3_$$4_,\
  $$(eval $$(call lu-def-addvar-inherited-ctx-master,$$1,$$2)) \
  $$(eval $$(call lu-def-addvar-inherited-ctx-global-index,$$1,$$3,$$4)) \
))

# Per target and per index/glossary variable
# $(2)$(EXT of $3)_$(4)_$(5)_VAR (source.dvi_INDEX_source_DEPENDS)
# 2: master name
# 3: flavor name
# 4: type (INDEX, GLOSS, ...)
# 5: index name
# Inherit from $2$(EXT of $3)_VAR $(2)_$(3)_$(4)_VAR
# (source.dvi_DEPENDS source_INDEX_source_DEPENDS)
$(eval $(call lu-define-addvar,-index,$$2$$(call lu-getvalue-flavor,EXT,$$3)_$$4_$$5_,index $$2$$(call lu-getvalue-flavor,EXT,$$3)/$$5[$$4],6,\
  $$2$$(call lu-getvalue-flavor,EXT,$$3)_ $$2_$$4_$$5_,\
  $$(eval $$(call lu-def-addvar-inherited-ctx,$$1,$$2,$$3)) \
  $$(eval $$(call lu-def-addvar-inherited-ctx-master-index,$$1,$$2,$$4,$$5)) \
))






define lu-setvar-global # 1:name 2:value
  _LU_$(1) ?= $(2)
  $$(eval $$(call lu-show-set-var,$(1),global,$(2)))
endef

define lu-setvar-flavor # 1:name 2:flavor 3:value
  _LU_FLAVOR_$(2)_$(1) ?= $(3)
  $$(eval $$(call lu-show-set-var,$(1),flavor $(2),$(3)))
endef

define lu-setvar-master # 1:name 2:master 3:value
  _LU_$(2)_$(1) ?= $(3)
  $$(eval $$(call lu-show-set-var,$(1),master $(2),$(3)))
endef

define lu-setvar # 1:name 2:master 3:flavor 4:value
  _LU_$(2)$$(call lu-getvalue-flavor,EXT,$(3))_$(1)=$(4)
  $$(eval $$(call lu-show-set-var,$(1),master/flavor $(2)/$(3),$(4)))
endef

define lu-getvalue # 1:name 2:master 3:flavor
$(call lu-show-readone-var,$(1),master/flavor $(2)/$(3),$(or \
	$(LU_$(2)$(call lu-getvalue-flavor,EXT,$(3))_$(1)), \
	$(TD_$(2)$(call lu-getvalue-flavor,EXT,$(3))_$(1)), \
	$(LU_$(2)_$(1)), \
	$($(2)_$(1)), \
	$(LU_FLAVOR_$(3)_$(1)), \
	$(LU_$(1)), \
	$($(1)), \
	$(_LU_$(2)$(call lu-getvalue-flavor,EXT,$(3))_$(1)), \
	$(_LU_$(2)_$(1)), \
	$(_LU_FLAVOR_$(3)_$(1)), \
	$(_LU_$(1))\
))
endef

define lu-getvalue-flavor # 1:name 2:flavor
$(call lu-show-readone-var,$(1),flavor $(2),$(or \
	$(LU_FLAVOR_$(2)_$(1)), \
	$(LU_$(1)), \
	$($(1)), \
	$(_LU_FLAVOR_$(2)_$(1)), \
	$(_LU_$(1))\
))
endef

define lu-getvalue-master # 1:name 2:master
$(call lu-show-readone-var,$(1),master $(2),$(or \
	$(LU_$(2)_$(1)), \
	$($(2)_$(1)), \
	$(LU_$(1)), \
	$($(1)), \
	$(_LU_$(2)_$(1)), \
	$(_LU_$(1))\
))
endef

define lu-getvalue-index # 1:name 2:master 3:flavor 4:type 5:indexname
$(call lu-show-readone-var,$(1),master/flavor/index $(2)/$(3)/[$(4)]$(5),$(or \
	$(LU_$(2)$(call lu-getvalue-flavor,EXT,$(3))_$(4)_$(5)_$(1)), \
	$(LU_$(2)_$(4)_$(5)_$(1)), \
	$(TD_$(2)$(call lu-getvalue-flavor,EXT,$(3))_$(4)_$(5)_$(1)), \
	$($(2)_$(4)_$(5)_$(1)), \
	$(LU_$(4)_$(5)_$(1)), \
	$($(4)_$(5)_$(1)), \
	$(LU_$(2)$(call lu-getvalue-flavor,EXT,$(3))_$(4)_$(1)), \
	$(LU_$(2)_$(4)_$(1)), \
	$($(2)_$(4)_$(1)), \
	$(LU_$(4)_$(1)), \
	$($(4)_$(1)), \
	$(LU_$(2)_$(1)), \
	$($(2)_$(1)), \
	$(LU_FLAVOR_$(3)_$(1)), \
	$(LU_$(1)), \
	$($(1)), \
	$(_LU_$(2)$(call lu-getvalue-flavor,EXT,$(3))_$(4)_$(5)_$(1)), \
	$(_LU_$(2)_$(4)_$(5)_$(1)), \
	$(_LU_$(4)_$(5)_$(1)), \
	$(_LU_$(2)$(call lu-getvalue-flavor,EXT,$(3))_$(4)_$(1)), \
	$(_LU_$(2)_$(4)_$(1)), \
	$(_LU_FLAVOR_$(3)_$(4)_$(1)), \
	$(_LU_$(4)_$(1)), \
	$(_LU_$(2)$(call lu-getvalue-flavor,EXT,$(3))_$(1)), \
	$(_LU_$(2)_$(1)), \
	$(_LU_FLAVOR_$(3)_$(1)), \
	$(_LU_$(1))\
))
endef

define lu-call-prog # 1:varname 2:master 3:flavor [4:index]
$(call lu-getvalue,$(1),$(2),$(3)) $(call lu-getvalues,$(1)_OPTIONS,$(2),$(3))
endef

define lu-call-prog-index # 1:varname 2:master 3:flavor 4:type 5:indexname
$(call lu-getvalue$(if $(4),-index),$(1),$(2),$(3),$(4),$(5)) \
	$(call lu-getvalues$(if $(4),-index),$(1)_OPTIONS,$(2),$(3),$(4),$(5))
endef

define lu-call-prog-flavor # 1:master 2:flavor
$(call lu-call-prog,$(call lu-getvalue,VARPROG,$(1),$(2)),$(1),$(2))
endef

#########################################################################
#########################################################################
#########################################################################
#########################################################################
##################                              #########################
##################     Global variables         #########################
##################                              #########################
#########################################################################
#########################################################################
#########################################################################
#########################################################################
#########################################################################

# Globals variables
$(eval $(call lu-setvar-global,LATEX,latex))
$(eval $(call lu-setvar-global,PDFLATEX,pdflatex))
$(eval $(call lu-setvar-global,LUALATEX,lualatex))
$(eval $(call lu-setvar-global,DVIPS,dvips))
$(eval $(call lu-setvar-global,DVIPDFM,dvipdfm))
$(eval $(call lu-setvar-global,BIBTEX,bibtex))
#$(eval $(call lu-setvar-global,MPOST,TEX="$(LATEX)" mpost))
$(eval $(call lu-setvar-global,FIG2DEV,fig2dev))
#$(eval $(call lu-setvar-global,SVG2DEV,svg2dev))
$(eval $(call lu-setvar-global,EPSTOPDF,epstopdf))
$(eval $(call lu-setvar-global,MAKEINDEX,makeindex))

# workaround the fact that $(shell ...) ignore locally exported variables
# get only the variables with plain names
_LU_MAKE_ENV := $(shell echo '$(.VARIABLES)' | awk -v RS=' ' '/^[a-zA-Z0-9]+$$/')
_LU_SHELL_EXPORT := $(foreach v,$(_LU_MAKE_ENV),$(v)='$($(v))')
_lu_run_kpsewhich=$(shell $(_LU_SHELL_EXPORT) kpsewhich -format $1 $2)

# Look first into the TDS (texmfscripts), then in PATH for our program
# At each location, we prefer with suffix than without
define _lu_which # VARNAME progname
 ifeq ($(origin _LU_$(1)_DEFAULT), undefined)
 _LU_$(1)_DEFAULT := $$(firstword $$(wildcard \
	$$(call _lu_run_kpsewhich,texmfscripts,$(2)) \
	$$(call _lu_run_kpsewhich,texmfscripts,$$(basename $(2))) \
	$$(foreach dir,$$(subst :, ,$$(PATH)), \
		$$(dir)/$(2) $$(dir)/$$(basename $(2))) \
	) $(2))
 export _LU_$(1)_DEFAULT
 _LU_$(1)_DEFAULT_OLD := $$(firstword $$(wildcard \
     $$(addprefix bin/,$(2) $$(basename $(2))) \
     $$(addprefix ./,$(2) $$(basename $(2)))))
 $$(if $$(filter-out $$(_LU_$(1)_DEFAULT), $$(_LU_$(1)_DEFAULT_OLD)),\
   $$(if $$(_lu_scripts_warnings),, \
     $$(eval _lu_scripts_warnings:=done) \
     $$(warning By default, this version of LaTeX-Make do not use \
     scripts in $$(dir $$(_LU_$(1)_DEFAULT_OLD)) anymore.) \
     $$(warning For example $$(_LU_$(1)_DEFAULT) is used instead of $$(_LU_$(1)_DEFAULT_OLD))\
     $$(warning If you want to keep the old behavior, add into your \
     Makefile something like:)\
     $$(warning export TEXMFSCRIPTS:=$$(dir $$(_LU_$(1)_DEFAULT_OLD))$$$$(addprefix :,$$$$(TEXMFSCRIPTS))::)))
 #$$(warning _LU_$(1)_DEFAULT=$$(_LU_$(1)_DEFAULT))
 endif
 $$(eval $$(call lu-setvar-global,$(1),$$(_LU_$(1)_DEFAULT)))
endef

$(eval $(call _lu_which,GENSUBFIG,gensubfig.py))
$(eval $(call _lu_which,FIGDEPTH,figdepth.py))
$(eval $(call _lu_which,GENSUBSVG,gensubfig.py))
$(eval $(call _lu_which,SVGDEPTH,svgdepth.py))
$(eval $(call _lu_which,SVG2DEV,svg2dev.py))
$(eval $(call _lu_which,LATEXFILTER,latexfilter.py))

# Rules to use to check if the build document (dvi or pdf) is up-to-date
# This can be overruled per document manually and/or automatically
#REBUILD_RULES ?= latex texdepends bibtopic bibtopic_undefined_references
$(eval $(call lu-addtovar-global,REBUILD_RULES,latex texdepends))

# Default maximum recursion level
$(eval $(call lu-setvar-global,MAX_REC,6))

#########################################################################
#########################################################################
#########################################################################
#########################################################################
##################                              #########################
##################          Flavors             #########################
##################                              #########################
#########################################################################
#########################################################################
#########################################################################
#########################################################################
#########################################################################

define lu-create-texflavor # 1:name 2:tex_prog 3:file_ext
			   # 4:master_cible 5:fig_extention_to_clean
  _LU_FLAVORS_DEFINED_TEX += $(1)
  $(eval $(call lu-setvar-flavor,VARPROG,$(1),$(2)))
  $(eval $(call lu-setvar-flavor,EXT,$(1),$(3)))
  $(eval $(call lu-setvar-flavor,TARGETNAME,$(1),$(4)))
  $(eval $(call lu-addtovar-flavor,CLEANFIGEXT,$(1),$(5)))
  $(eval $(call lu-addtovar-flavor,CLEANSVGEXT,$(1),$(5)))
endef

define lu-create-dviflavor # 1:name 2:dvi_prog 3:file_ext 
			   # 4:master_cible 5:tex_flavor_depend
  $$(eval $$(call lu-define-flavor,$(5)))
  _LU_FLAVORS_DEFINED_DVI += $(1)
  $(eval $(call lu-setvar-flavor,VARPROG,$(1),$(2)))
  $(eval $(call lu-setvar-flavor,EXT,$(1),$(3)))
  $(eval $(call lu-setvar-flavor,TARGETNAME,$(1),$(4)))
  $(eval $(call lu-setvar-flavor,DEPFLAVOR,$(1),$(5)))
endef

define lu-create-flavor # 1:name 2:type 3..7:options
  $$(if $$(filter $(1),$(_LU_FLAVORS_DEFINED)), \
	$$(call lu-show-flavors,Flavor $(1) already defined), \
	$$(call lu-show-flavors,Creating flavor $(1) ($(2))) \
	$$(eval $$(call lu-create-$(2)flavor,$(1),$(3),$(4),$(5),$(6),$(7))))
endef

define lu-define-flavor # 1:name
  $$(eval $$(call lu-define-flavor-$(1)))
endef

define lu-flavor-rules # 1:name
 $$(call lu-show-flavors,Defining rules for flavor $(1))
 $$(if $$(call lu-getvalue-flavor,TARGETNAME,$(1)), \
 $$(call lu-getvalue-flavor,TARGETNAME,$(1)): \
	$$(call lu-getvalues-flavor,TARGETS,$(1)))
 $$(if $$(call lu-getvalue-flavor,TARGETNAME,$(1)), \
 .PHONY: $$(call lu-getvalue-flavor,TARGETNAME,$(1)))
endef

define lu-define-flavor-DVI #
  $$(eval $$(call lu-create-flavor,DVI,tex,LATEX,.dvi,dvi,\
	.pstex_t .pstex))
endef

define lu-define-flavor-PDF #
  $$(eval $$(call lu-create-flavor,PDF,tex,PDFLATEX,.pdf,pdf,\
	.pdftex_t .$$(_LU_PDFTEX_EXT)))
endef

define lu-define-flavor-LUALATEX #
  $$(eval $$(call lu-create-flavor,LUALATEX,tex,LUALATEX,.pdf,pdf,\
	.pdftex_t .$$(_LU_PDFTEX_EXT)))
endef

define lu-define-flavor-PS #
  $$(eval $$(call lu-create-flavor,PS,dvi,DVIPS,.ps,ps,DVI))
endef

define lu-define-flavor-DVIPDF #
  $$(eval $$(call lu-create-flavor,DVIPDF,dvi,DVIPDFM,.pdf,pdf,DVI))
endef

$(eval $(call lu-addtovar-global,FLAVORS,PDF PS))

#########################################################################
#########################################################################
#########################################################################
#########################################################################
##################                              #########################
##################          Masters             #########################
##################                              #########################
#########################################################################
#########################################################################
#########################################################################
#########################################################################
#########################################################################

define _lu-do-latex # 1:master 2:flavor 3:source.tex 4:ext(.dvi/.pdf)
  exec 3>&1; \
  run() { \
	printf "Running:" 1>&3 ; \
	for arg; do \
		printf "%s" " '$$arg'" 1>&3 ; \
	done ; echo 1>&3 ; \
	"$$@" ; \
  }; \
  doit() { \
	$(RM) -v "$(1)$(4)_FAILED"  \
		"$(1)$(4)_NEED_REBUILD" \
		"$(1)$(4).mk" ;\
		( 	echo X | \
			run $(call lu-call-prog-flavor,$(1),$(2)) \
				--interaction errorstopmode \
				--jobname "$(1)" \
	'\RequirePackage[extension='"$(4)"']{texdepends}\input'"{$(3)}" || \
			touch "$(1)$(4)_FAILED" ; \
			if grep -sq '^! LaTeX Error:' "$(1).log" ; then \
				touch "$(1)$(4)_FAILED" ; \
			fi \
		) | $(call lu-call-prog,LATEXFILTER,$(1),$(2)) ; \
	NO_TEXDEPENDS_FILE=0 ;\
	if [ ! -f "$(1)$(4).mk" ]; then \
		NO_TEXDEPENDS_FILE=1 ;\
	fi ;\
	sed -e 's,\\openout[0-9]* = \([^`].*\),TD_$(1)$(4)_OUTPUTS += \1,p;s,\\openout[0-9]* = `\(.*\)'"'.,TD_$(1)$(4)_OUTPUTS += \1,p;d" \
		"$(1).log" >> "$(1)$(4).mk" ;\
	if [ -f "$(1)$(4)_FAILED" ]; then \
		echo "*************************************" ;\
		echo "Building $(1)$(4) fails" ;\
		echo "*************************************" ;\
		echo "Here are the last lines of the log file" ;\
		echo "If this is not enought, try to" ;\
		echo "call 'make' with 'VERB=verbose' option" ;\
		echo "*************************************" ;\
		echo "==> Last lines in $(1).log <==" ; \
		sed -e '/^[?] X$$/,$$d' \
		    -e '/^Here is how much of TeX'"'"'s memory you used:$$/,$$d' \
			< "$(1).log" | tail -n 20; \
		return 1; \
	fi; \
	if [ "$$NO_TEXDEPENDS_FILE" = 1 ]; then \
		echo "*************************************" ;\
		echo "texdepends does not seems be loaded" ;\
		echo "Either your (La)TeX installation is wrong or you found a bug." ;\
		echo "If so, please, report it (with the result of shell command 'kpsepath tex')";\
		echo "Aborting compilation" ;\
		echo "*************************************" ;\
		touch "$(1)$(4)_FAILED" ; \
		return 1 ;\
	fi ;\
    }; doit
endef

.PHONY: clean-build-fig

##########################################################
define lu-master-texflavor-index-vars # MASTER FLAVOR TYPE INDEX ext(.dvi/.pdf)
 $$(call lu-show-rules,Setting flavor index vars for $(1)/$(2)/[$(3)]$(4))
 $$(eval $$(call lu-addtovar,DEPENDS,$(1),$(2), \
    $$(call lu-getvalue-index,TARGET,$(1),$(2),$(3),$(4))))
 $$(eval $$(call lu-addtovar,WATCHFILES,$(1),$(2), \
    $$(call lu-getvalue-index,SRC,$(1),$(2),$(3),$(4))))
endef ####################################################
define lu-master-texflavor-index-rules # MASTER FLAVOR TYPE INDEX ext(.dvi/.pdf)
 $$(call lu-show-rules,Setting flavor index rules for $(1)/$(2)/[$(3)]$(4))
 $$(if $$(_LU_DEF_IND_$$(call lu-getvalue-index,TARGET,$(1),$(2),$(3),$(4))), \
   $$(call lu-show-rules,=> Skipping: already defined in flavor $$(_LU_DEF_IND_$$(call lu-getvalue-index,TARGET,$(1),$(2),$(3),$(4)))), \
   $$(eval $$(call _lu-master-texflavor-index-rules\
	,$(1),$(2),$(3),$(4),$(5),$$(call lu-getvalue-index,TARGET,$(1),$(2),$(3),$(4)))))
endef
define _lu-master-texflavor-index-rules # MASTER FLAVOR TYPE INDEX ext TARGET
 $(6): \
    $$(call lu-getvalue-index,SRC,$(1),$(2),$(3),$(4)) \
    $$(wildcard $$(call lu-getvalue-index,STYLE,$(1),$(2),$(3),$(4)))
	$$(COMMON_PREFIX)$$(call lu-call-prog-index,MAKEINDEX,$(1),$(2),$(3),$(4)) \
	  $$(addprefix -s ,$$(call lu-getvalue-index,STYLE,$(1),$(2),$(3),$(4))) \
	  -o $$@ $$<
 _LU_DEF_IND_$(6)=$(2)
 clean::
	$$(call lu-clean,$$(call lu-getvalue-index,TARGET,$(1),$(2),$(3),$(4)) \
		$$(addsuffix .ilg,$$(basename \
			$$(call lu-getvalue-index,SRC,$(1),$(2),$(3),$(4)))))
endef ####################################################
define lu-master-texflavor-index # MASTER FLAVOR INDEX ext(.dvi/.pdf)
 $$(eval $$(call lu-master-texflavor-index-vars,$(1),$(2),$(3),$(4)))
 $$(eval $$(call lu-master-texflavor-index-rules,$(1),$(2),$(3),$(4)))
endef
##########################################################

##########################################################
define lu-master-texflavor-vars # MASTER FLAVOR ext(.dvi/.pdf)
 $$(call lu-show-rules,Setting flavor vars for $(1)/$(2))
 -include $(1)$(3).mk
 $$(eval $$(call lu-addtovar,DEPENDS,$(1),$(2), \
               $$(call lu-getvalues,FIGURES,$(1),$(2)) \
               $$(call lu-getvalues,BIBFILES,$(1),$(2)) \
   $$(wildcard $$(call lu-getvalues,INPUTS,$(1),$(2))) \
   $$(wildcard $$(call lu-getvalues,BIBSTYLES,$(1),$(2))) \
               $$(call lu-getvalues,BBLFILES,$(1),$(2))\
 ))

 $$(eval $$(call lu-addtovar-flavor,TARGETS,$(2),$(1)$(3)))

 $$(eval $$(call lu-addtovar,GPATH,$(1),$(2), \
     $$(subst },,$$(subst {,,$$(subst }{, ,\
	$$(call lu-getvalue,GRAPHICSPATH,$(1),$(2)))))))

 $$(if $$(sort $$(call lu-getvalues,SUBFIGS,$(1),$(2))), \
	$$(eval include $$(addsuffix .mk,$$(sort \
		$$(call lu-getvalues,SUBFIGS,$(1),$(2))))))

 $$(eval $$(call lu-addtovar,WATCHFILES,$(1),$(2), \
	$$(filter %.aux, $$(call lu-getvalues,OUTPUTS,$(1),$(2)))))

 $$(foreach type,$$(call lu-getvalues,INDEXES,$(1),$(2)), \
   $$(foreach index,$$(call lu-getvalues,INDEXES_$$(type),$(1),$(2)), \
    $$(eval $$(call lu-master-texflavor-index-vars,$(1),$(2),$$(type),$$(index),$(3)))))
endef ####################################################
define lu-master-texflavor-rules # MASTER FLAVOR ext(.dvi/.pdf)
 $$(call lu-show-rules,Defining flavor rules for $(1)/$(2))
 $$(call lu-getvalues,BBLFILES,$(1),$(2)): \
	$$(sort             $$(call lu-getvalues,BIBFILES,$(1),$(2)) \
		$$(wildcard $$(call lu-getvalues,BIBSTYLES,$(1),$(2))))
 $(1)$(3): %$(3): \
   $$(filter-out $$(call lu-getvalues,DEPENDS_EXCLUDE,$(1),$(2)), \
     $$(call lu-getvalues,DEPENDS,$(1),$(2)) \
     $$(call lu-getvalues,REQUIRED,$(1),$(2))) \
   $$(if $$(wildcard $(1)$(3)_FAILED),LU_FORCE,) \
   $$(if $$(wildcard $(1)$(3)_NEED_REBUILD),LU_FORCE,) \
   $$(if $$(wildcard $(1)$(3)_NEED_REBUILD_IN_PROGRESS),LU_FORCE,)
	$$(if $$(filter-out $$(LU_REC_LEVEL),$$(call lu-getvalue,MAX_REC,$(1),$(2))),, \
		$$(warning *********************************) \
		$$(warning *********************************) \
		$$(warning *********************************) \
		$$(warning Stopping generation of $$@) \
		$$(warning I got max recursion level $$(call lu-getvalue,MAX_REC,$(1),$(2))) \
		$$(warning Set LU_$(1)_$(2)_MAX_REC, LU_MAX_REC_$(1) or LU_MAX_REC if you need it) \
		$$(warning *********************************) \
		$$(warning *********************************) \
		$$(warning *********************************) \
		$$(error Aborting generation of $$@))
	$$(MAKE) LU_REC_MASTER="$(1)" LU_REC_FLAVOR="$(2)" LU_REC_TARGET="$$@"\
		LU_WATCH_FILES_SAVE
	$$(COMMON_PREFIX)$$(call _lu-do-latex\
		,$(1),$(2),$$(call lu-getvalue-master,MAIN,$(1)),$(3))
	$$(MAKE) LU_REC_MASTER="$(1)" LU_REC_FLAVOR="$(2)" LU_REC_TARGET="$$@"\
		LU_WATCH_FILES_RESTORE
	$$(MAKE) LU_REC_MASTER="$(1)" LU_REC_FLAVOR="$(2)" LU_REC_TARGET="$$@"\
		$(1)$(3)_NEED_REBUILD
ifneq ($(LU_REC_TARGET),)
 $(1)$(3)_NEED_REBUILD_IN_PROGRESS:
	$$(COMMON_HIDE)touch $(1)$(3)_NEED_REBUILD_IN_PROGRESS
 $$(addprefix LU_rebuild_,$$(call lu-getvalues,REBUILD_RULES,$(1),$(2))): \
	$(1)$(3)_NEED_REBUILD_IN_PROGRESS
.PHONY: $(1)$(3)_NEED_REBUILD
 $(1)$(3)_NEED_REBUILD: \
    $(1)$(3)_NEED_REBUILD_IN_PROGRESS \
    $$(addprefix LU_rebuild_,$$(call lu-getvalues,REBUILD_RULES,$(1),$(2)))
	$$(COMMON_HIDE)$(RM) $(1)$(3)_NEED_REBUILD_IN_PROGRESS
	$$(COMMON_HIDE)if [ -f "$(1)$(3)_NEED_REBUILD" ];then\
		echo "********************************************" ;\
		echo "*********** New build needed ***************" ;\
		echo "********************************************" ;\
		cat "$(1)$(3)_NEED_REBUILD" ; \
		echo "********************************************" ;\
	fi
	$$(MAKE) LU_REC_LEVEL=$$(shell expr $$(LU_REC_LEVEL) + 1) \
		$$(LU_REC_TARGET)
endif
 clean-build-fig::
	$$(call lu-clean,$$(foreach fig, \
	   $$(basename $$(wildcard $$(filter %.fig, \
			$$(call lu-getvalues,FIGURES,$(1),$(2))))), \
	   $$(addprefix $$(fig),$$(call lu-getvalues-flavor,CLEANFIGEXT,$(2)))))
	$$(call lu-clean,$$(foreach svg, \
	   $$(basename $$(wildcard $$(filter %.svg, \
			$$(call lu-getvalues,FIGURES,$(1),$(2))))), \
	   $$(addprefix $$(svg),$$(call lu-getvalues-flavor,CLEANSVGEXT,$(2)))))
 clean:: clean-build-fig
	$$(call lu-clean,$$(call lu-getvalues,OUTPUTS,$(1),$(2)) \
		$$(call lu-getvalues,BBLFILES,$(1),$(2)) \
		$$(addsuffix .mk,$$(call lu-getvalues,SUBFIGS,$(1),$(2))) \
	    $$(patsubst %.bbl,%.blg,$$(call lu-getvalues,BBLFILES,$(1),$(2))))
	$$(call lu-clean,$$(wildcard $(1).log))
 distclean::
	$$(call lu-clean,$$(wildcard $(1)$(3) $(1)$(3)_FAILED \
		$(1)$(3)_NEED_REBUILD $(1)$(3)_NEED_REBUILD_IN_PROGRESS))
 $$(foreach type,$$(call lu-getvalues,INDEXES,$(1),$(2)), \
   $$(foreach index,$$(call lu-getvalues,INDEXES_$$(type),$(1),$(2)), \
    $$(eval $$(call lu-master-texflavor-index-rules,$(1),$(2),$$(type),$$(index),$(3)))))
endef ####################################################
define lu-master-texflavor # MASTER FLAVOR ext(.dvi/.pdf)
 $$(eval $$(call lu-master-texflavor-vars,$(1),$(2),$(3)))
 $$(eval $$(call lu-master-texflavor-rules,$(1),$(2),$(3)))
endef
##########################################################

##########################################################
define lu-master-dviflavor-vars # MASTER FLAVOR ext(.ps)
 $$(call lu-show-rules,Setting flavor vars for \
	$(1)/$(2)/$$(call lu-getvalue-flavor,DEPFLAVOR,$(2)))
# $$(eval $$(call lu-addvar,VARPROG,$(1),$(2)))
# $$(eval $$(call lu-addvar,$$(call lu-getvalue,VARPROG,$(1),$(2)),$(1),$(2)))
 $$(eval $$(call lu-addtovar-flavor,TARGETS,$(2),$(1)$(3)))
endef ####################################################
define lu-master-dviflavor-rules # MASTER FLAVOR ext(.ps)
 $$(call lu-show-rules,Defining flavor rules for \
	$(1)/$(2)/$$(call lu-getvalue-flavor,DEPFLAVOR,$(2)))
 $(1)$(3): %$(3): %$$(call lu-getvalue-flavor,EXT,$$(call lu-getvalue-flavor,DEPFLAVOR,$(2)))
	$$(call lu-call-prog-flavor,$(1),$(2))	-o $$@ $$<
 distclean::
	$$(call lu-clean,$$(wildcard $(1)$(3)))
endef ####################################################
define lu-master-dviflavor # MASTER FLAVOR ext(.ps)
 $$(eval $$(call lu-master-dviflavor-vars,$(1),$(2),$(3)))
 $$(eval $$(call lu-master-dviflavor-rules,$(1),$(2),$(3)))
endef
##########################################################

##########################################################
define lu-master-vars # MASTER
 $$(call lu-show-rules,Setting vars for $(1))
 $$(eval $$(call lu-setvar-master,MAIN,$(1),$(1).tex))
 $$(eval $$(call lu-addtovar-master,DEPENDS,$(1),\
	$$(call lu-getvalue-master,MAIN,$(1))))
 _LU_$(1)_DVI_FLAVORS=$$(filter $$(_LU_FLAVORS_DEFINED_DVI),\
	$$(sort $$(call lu-getvalues-master,FLAVORS,$(1))))
 _LU_$(1)_TEX_FLAVORS=$$(filter $$(_LU_FLAVORS_DEFINED_TEX),\
	$$(sort $$(call lu-getvalues-master,FLAVORS,$(1)) \
		$$(LU_REC_FLAVOR) \
	$$(foreach dvi,$$(call lu-getvalues-master,FLAVORS,$(1)), \
		$$(call lu-getvalue-flavor,DEPFLAVOR,$$(dvi)))))
 $$(foreach flav,$$(_LU_$(1)_TEX_FLAVORS), $$(eval $$(call \
	lu-master-texflavor-vars,$(1),$$(flav),$$(call lu-getvalue-flavor,EXT,$$(flav)))))
 $$(foreach flav,$$(_LU_$(1)_DVI_FLAVORS), $$(eval $$(call \
	lu-master-dviflavor-vars,$(1),$$(flav),$$(call lu-getvalue-flavor,EXT,$$(flav)))))
endef ####################################################
define lu-master-rules # MASTER
 $$(call lu-show-rules,Defining rules for $(1))
 $$(foreach flav,$$(_LU_$(1)_TEX_FLAVORS), $$(eval $$(call \
	lu-master-texflavor-rules,$(1),$$(flav),$$(call lu-getvalue-flavor,EXT,$$(flav)))))
 $$(foreach flav,$$(_LU_$(1)_DVI_FLAVORS), $$(eval $$(call \
	lu-master-dviflavor-rules,$(1),$$(flav),$$(call lu-getvalue-flavor,EXT,$$(flav)))))
endef ####################################################
define lu-master # MASTER
 $$(eval $$(call lu-master-vars,$(1)))
 $$(eval $$(call lu-master-rules,$(1)))
endef
##########################################################

#$(warning $(call LU_RULES,example))
$(eval $(call lu-addtovar-global,MASTERS,\
	$$(shell grep -l '\\documentclass' *.tex 2>/dev/null | sed -e 's/\.tex$$$$//')))
ifneq ($(LU_REC_TARGET),)
_LU_DEF_MASTERS = $(LU_REC_MASTER)
_LU_DEF_FLAVORS = $(LU_REC_FLAVOR) $(FLAV_DEPFLAVOR_$(LU_REC_FLAVOR))
else
_LU_DEF_MASTERS = $(call lu-getvalues-global,MASTERS)
_LU_DEF_FLAVORS = $(sort $(foreach master,$(_LU_DEF_MASTERS),\
	$(call lu-getvalues-master,FLAVORS,$(master))))
endif

$(foreach flav, $(_LU_DEF_FLAVORS), $(eval $(call lu-define-flavor,$(flav))))
$(foreach master, $(_LU_DEF_MASTERS), $(eval $(call lu-master-vars,$(master))))
$(foreach flav, $(_LU_FLAVORS_DEFINED), $(eval $(call lu-flavor-rules,$(flav))))
$(foreach master, $(_LU_DEF_MASTERS), $(eval $(call lu-master-rules,$(master))))

##################################################################""
# Gestion des subfigs

%<<MAKEFILE
%.subfig.mk: %.subfig %.fig
	$(COMMON_PREFIX)$(call lu-call-prog,GENSUBFIG) \
		-p '$$(COMMON_PREFIX)$(call lu-call-prog,FIGDEPTH) < $$< > $$@' \
		-s $*.subfig $*.fig < $^ > $@
%MAKEFILE

%<<MAKEFILE
%.subfig.mk: %.subfig %.svg
	$(COMMON_PREFIX)$(call lu-call-prog,GENSUBSVG) \
		-p '$$(COMMON_PREFIX)$(call lu-call-prog,SVGDEPTH) < $$< > $$@' \
		-s $*.subfig $*.svg < $^ > $@
%MAKEFILE

clean::
	$(call lu-clean,$(FIGS2CREATE_LIST))
	$(call lu-clean,$(FIGS2CREATE_LIST:%.fig=%.pstex))
	$(call lu-clean,$(FIGS2CREATE_LIST:%.fig=%.pstex_t))
	$(call lu-clean,$(FIGS2CREATE_LIST:%.fig=%.$(_LU_PDFTEX_EXT)))
	$(call lu-clean,$(FIGS2CREATE_LIST:%.fig=%.pdftex_t))
	$(call lu-clean,$(FIGS2CREATE_LIST:%.svg=%.pstex))
	$(call lu-clean,$(FIGS2CREATE_LIST:%.svg=%.pstex_t))
	$(call lu-clean,$(FIGS2CREATE_LIST:%.svg=%.$(_LU_PDFTEX_EXT)))
	$(call lu-clean,$(FIGS2CREATE_LIST:%.svg=%.pdftex_t))

.PHONY: LU_FORCE clean distclean
LU_FORCE:
	@echo "Previous compilation failed. Rerun needed"

#$(warning $(MAKEFILE))

distclean:: clean

%<<MAKEFILE
%.eps: %.fig
	$(COMMON_PREFIX)$(call lu-call-prog,FIG2DEV) -L eps $< $@

%.pdf: %.fig
	$(COMMON_PREFIX)$(call lu-call-prog,FIG2DEV) -L pdf $< $@

%.pstex: %.fig
	$(COMMON_PREFIX)$(call lu-call-prog,FIG2DEV) -L pstex $< $@

%.pstex: %.svg
	$(COMMON_PREFIX)$(call lu-call-prog,SVG2DEV) -L pstex $< $@


.PRECIOUS: %.pstex
%.pstex_t: %.fig %.pstex
	$(COMMON_PREFIX)$(call lu-call-prog,FIG2DEV) -L pstex_t -p $*.pstex $< $@

%.pstex_t: %.svg %.pstex
	$(COMMON_PREFIX)$(call lu-call-prog,SVG2DEV) -L pstex_t -p $*.pstex $< $@


%.$(_LU_PDFTEX_EXT): %.fig
	$(COMMON_PREFIX)$(call lu-call-prog,FIG2DEV) -L pdftex $< $@

%.$(_LU_PDFTEX_EXT): %.svg
	$(COMMON_PREFIX)$(call lu-call-prog,SVG2DEV) -L pdftex $< $@

.PRECIOUS: %.$(_LU_PDFTEX_EXT)
%.pdftex_t: %.fig %.$(_LU_PDFTEX_EXT)
	$(COMMON_PREFIX)$(call lu-call-prog,FIG2DEV) -L pdftex_t -p $*.$(_LU_PDFTEX_EXT) $< $@

%.pdftex_t: %.svg %.$(_LU_PDFTEX_EXT)
	$(COMMON_PREFIX)$(call lu-call-prog,SVG2DEV) -L pdftex_t -p $*.$(_LU_PDFTEX_EXT) $< $@

%.pdf: %.eps
	$(COMMON_PREFIX)$(call lu-call-prog,EPSTOPDF) --filter < $< > $@
%MAKEFILE

#########################################################################
# Les flavors
LU_REC_LEVEL ?= 1
ifneq ($(LU_REC_TARGET),)
export LU_REC_FLAVOR
export LU_REC_MASTER
export LU_REC_TARGET
export LU_REC_LEVEL
LU_REC_LOGFILE=$(LU_REC_MASTER).log
LU_REC_GENFILE=$(LU_REC_MASTER)$(call lu-getvalue-flavor,EXT,$(LU_REC_FLAVOR))

lu-rebuild-head=$(info *** Checking rebuild with rule '$(subst LU_rebuild_,,$@)')
lu-rebuild-needed=echo $(1) >> "$(LU_REC_GENFILE)_NEED_REBUILD" ;

.PHONY: $(addprefix LU_rebuild_,latex texdepends bibtex)
LU_rebuild_latex:
	$(call lu-rebuild-head)
	$(COMMON_HIDE)if grep -sq 'Rerun to get'\
		"$(LU_REC_LOGFILE)" ; then \
		$(call lu-rebuild-needed\
		,"$@: new run needed (LaTeX message 'Rerun to get...')") \
	fi

LU_rebuild_texdepends:
	$(call lu-rebuild-head)
	$(COMMON_HIDE)if grep -sq '^Package texdepends Warning: .* Check dependencies again.$$'\
		"$(LU_REC_LOGFILE)" ; then \
		$(call lu-rebuild-needed,"$@: new depends required") \
	fi

LU_rebuild_bibtopic:
	$(call lu-rebuild-head)
%</makefile>
%    \end{macrocode}
%  This part is not needed: already checked with the |lu_rebuild_latex| rule
%    \begin{macrocode}
%<*notused>
	$(COMMON_HIDE)if grep -sq 'Rerun to get indentation of bibitems right'\
		"$(LU_REC_LOGFILE)" ; then \
		$(call lu-rebuild-needed,"$@: new run needed") \
	fi
	$(COMMON_HIDE)if grep -sq 'Rerun to get cross-references right'\
		"$(LU_REC_LOGFILE)" ; then \
		$(call lu-rebuild-needed,"$@: new run needed") \
	fi
%</notused>
%<*makefile>
	$(COMMON_HIDE)sed -e '/^Package bibtopic Warning: Please (re)run BibTeX on the file(s):$$/,/^(bibtopic) *and after that rerun LaTeX./{s/^(bibtopic) *\([^ ]*\)$$/\1/p};d' \
				"$(LU_REC_LOGFILE)" | while read file ; do \
		touch $$file.aux ; \
		$(call lu-rebuild-needed,"bibtopic: $$file.bbl outdated") \
	done

LU_rebuild_bibtopic_undefined_references:
	$(call lu-rebuild-head)
	$(COMMON_HIDE)if grep -sq 'There were undefined references'\
		"$(MASTER_$(LU_REC_MASTER)).log" ; then \
		$(call lu-rebuild-needed,"$@: new run needed") \
	fi

.PHONY: LU_WATCH_FILES_SAVE LU_WATCH_FILES_RESTORE
LU_WATCH_FILES_SAVE:
	$(COMMON_HIDE)$(foreach file, $(sort \
		$(call lu-getvalues,WATCHFILES,$(LU_REC_MASTER),$(LU_REC_FLAVOR))), \
	    $(call lu-save-file,$(file),$(file).orig);)

LU_WATCH_FILES_RESTORE:
	$(COMMON_HIDE)$(foreach file, $(sort \
		$(call lu-getvalues,WATCHFILES,$(LU_REC_MASTER),$(LU_REC_FLAVOR))), \
	    $(call lu-cmprestaure-file,"$(file)","$(file).orig",\
		echo "New $(file) file" >> $(LU_REC_GENFILE)_NEED_REBUILD;\
		);)

endif

%<<MAKEFILE
%.bbl: %.aux
	$(COMMON_PREFIX)$(call lu-call-prog,BIBTEX) $*
%MAKEFILE

_LaTeX_Make_GROUPS=texmfscripts tex
_LaTeX_Make_texmfscripts = LaTeX.mk figdepth.py gensubfig.py svg2dev.py svgdepth.py latexfilter.py
_LaTeX_Make_texmfscripts_DIR = scripts/latex-make
_LaTeX_Make_tex = figlatex.sty pdfswitch.sty texdepends.sty texgraphicx.sty
_LaTeX_Make_tex_DIR = tex/latex/latex-make

.PHONY: LaTeX-Make-local-install LaTeX-Make-local-uninstall

LaTeX-Make-local-uninstall::
	$(if $(TEXMF_INSTALL_ROOT_DIR),,\
	$(error TEXMF_INSTALL_ROOT_DIR must be set when calling LaTeX-Make-local-uninstall))
	$(foreach g,$(_LaTeX_Make_GROUPS),\
		$(foreach f,$(_LaTeX_Make_$(g)), \
			$(LU_RM) $(TEXMF_INSTALL_ROOT_DIR)/$(_LaTeX_Make_$(g)_DIR)/$f && \
		) (rmdir $(TEXMF_INSTALL_ROOT_DIR)/$(_LaTeX_Make_$(g)_DIR) || true) && \
	) $(LU_RM) LaTeX.mk

LU_INSTALL_PKSEWHICH?=env -u TEXMFHOME kpsewhich
LaTeX-Make-local-install::
	$(if $(TEXMF_INSTALL_ROOT_DIR),,\
	$(error TEXMF_INSTALL_ROOT_DIR must be set when calling LaTeX-Make-local-install))
	$(if $(filter texmf,$(notdir $(TEXMF_INSTALL_ROOT_DIR))),,\
	$(if $(FORCE),,\
	$(warning TEXMF_INSTALL_ROOT_DIR does not end with 'texmf')\
	$(error Use FORCE=1 if you really want to use this value of TEXMF_INSTALL_ROOT_DIR)))
	$(foreach g,$(_LaTeX_Make_GROUPS),\
		mkdir -p $(TEXMF_INSTALL_ROOT_DIR)/$(_LaTeX_Make_$(g)_DIR) && \
		$(foreach f,$(_LaTeX_Make_$(g)), \
			$(LU_CP) -v $$($(LU_INSTALL_PKSEWHICH) -format $g $f)  $(TEXMF_INSTALL_ROOT_DIR)/$(_LaTeX_Make_$(g)_DIR) && \
	)) echo "Installation into $(TEXMF_INSTALL_ROOT_DIR) done."
	@echo "==> You must ensure your TEXMFHOME contains this path <=="

%</makefile>
%    \end{macrocode}
% \subsection{figdepth}
%
%    \begin{macrocode}
%<*figdepth>
#!/usr/bin/env python3
#coding=utf8

"""

stdin : the original xfig file
stdout : the output xfig file
args : all depths we want to keep

"""

from __future__ import print_function
import optparse
import os.path
import sys

def main():
    parser = optparse.OptionParser()
    (options, args) = parser.parse_args()

    depths_to_keep = set()
    for arg in args:
        depths_to_keep.add(arg)

    comment = ''
    display = True
    def show(depth, line):
        if depth in depths_to_keep:
            print(comment+line, end='')
            return True
        else:
            return False
    for line in sys.stdin:
        if line[0] == '#':
            comment += line
            continue
        if line[0] in "\t ":
            if display:
                print(line, end='')
        else:
            Fld = line.split(' ', 9999)
            if not Fld[0] or Fld[0] not in ('1', '2', '3', '4', '5'):
                print(comment+line, end='')
                display = True
            elif Fld[0] == '4':
                display = show(Fld[3], line)
            else:
                display = show(Fld[6], line)
            comment = ''

if __name__ == "__main__":
    main()
%</figdepth>
%    \end{macrocode}
% \subsection{gensubfig}
%    \begin{macrocode}
%<*gensubfig>
#!/usr/bin/env python3
#coding=utf8

"""

Arguments passes :
    - fichier image (image.fig ou image.svg)
    - -s fichier subfig (image.subfig)
    - -p chemin du script pour generer les sous-images (svgdepth.py ou figdepth.py)

Sortie standard :
    - makefile pour creer les sous-images (au format .fig ou .svg), et pour les supprimer

"""

from __future__ import print_function
from optparse import OptionParser
import os.path

def main():
    parser = OptionParser(usage='usage: %prog [options] svg file', description='Creates a\
Makefile generating subfigures using figdepth.py or svgdepth.py')
    parser.add_option("-s", "--subfig", dest="subfig", help="subfig file")
    parser.add_option("-p", "--depth", dest="depth", help="full path of depth script")
    (options, args) = parser.parse_args()
    if len(args) < 1:
        parser.error("incorrect number of arguments")
    if not options.subfig:
        parser.error("no subfig file specified")
    if not options.depth:
        parser.error("no depth script specified")

    (root, ext) = os.path.splitext(args[0])
    sf_name = options.subfig
    ds_name = options.depth
    varname = '%s_FIGS' % root.upper()

    subfigs = []
    for line in open(options.subfig, 'r'):
        t = line.find('#') # looking for comments
        if t > -1: line = line[0:t] # remove comments...
        line = line.strip() #remove blank chars
        if line == '': continue
        subfigs.append(line)

    count = 1
    for subfig in subfigs:
        print("%s_%d%s: %s%s %s" % (root, count, ext, root, ext, sf_name))
        print("\t%s %s" % (ds_name, subfig))
        print("")
        count += 1
    print("%s := $(foreach n, " % varname, end='')
    count = 1
    for subfig in subfigs:
        print('%d ' % count, end='')
        count += 1
    print(", %s_$(n)%s)" % (root, ext))
    print("FILES_TO_DISTCLEAN += $(%s)" % varname)
    print("FIGS2CREATE_LIST += $(%s)" % varname)
    print("$(TEMPORAIRE): $(%s)" % varname)

if __name__ == "__main__":
    main()
%</gensubfig>
%    \end{macrocode}
% \subsection{svg2dev}
%    \begin{macrocode}
%<*svg2dev>
#!/usr/bin/env python3
#coding=utf8

from optparse import OptionParser
import shutil
import os
import subprocess

svg2eps = 'inkscape %s -C --export-filename=%s.eps --export-type=eps --export-latex'
svg2pdf = 'inkscape %s -C --export-filename=%s.pdf --export-type=pdf --export-latex'

def create_image(input_filename, output_filename, mode, ext):
    subprocess.Popen(mode % (input_filename, output_filename),
        stdout=subprocess.PIPE, shell=True).communicate()[0]

    o_ext = output_filename + '.' + ext
    o = output_filename
    o_ext_tex = output_filename + '.' + ext + '_tex'
    o_t = output_filename + '_t'

    shutil.move(o_ext, o)

    fin = open(o_ext_tex, 'r')
    fout = open(o_t, 'w')

    #\includegraphics[width=\unitlength,page=1]{logo.pdftex}
    for line in fin:
        # FIXME: be more conservative in the replacement
        # (in case '{'+o_ext+'}' appeares somewhere else)
        out = line.replace('{'+os.path.basename(o_ext)+'}', '{'+os.path.basename(o)+'}')
        fout.write(out)

    fin.close()
    fout.close()
    os.remove(o_ext_tex)

def main():
    parser = OptionParser()
    parser.add_option("-L", "--format", dest="outputFormat",
        metavar="FORMAT", help="output format", default="spstex")
    parser.add_option("-p", "--portrait", dest="portrait", help="dummy arg")
    (options, args) = parser.parse_args()
    if len(args) != 2: return
    (input_filename, output_filename) = args
    fmt = options.outputFormat
    portrait = options.portrait

    if fmt == 'eps':
        create_image(input_filename, output_filename, svg2eps, 'eps')
    elif fmt == 'spstex' or fmt == 'pstex':
        create_image(input_filename, output_filename, svg2eps, 'eps')
    elif fmt == 'spstex_t' or fmt == 'pstex_t':
        pass
    elif fmt == 'spdftex' or fmt == 'pdftex':
        create_image(input_filename, output_filename, svg2pdf, 'pdf')
    elif fmt == 'spdftex_t' or fmt == 'pdftex_t':
        pass

if __name__ == "__main__":
    main()

%</svg2dev>
%    \end{macrocode}
% \subsection{latexfilter}
% |latexfilter.py| is a small python program that hides most of the output
% of \TeX/\LaTeX{} output. It only display info, warnings, errors
% and underfull/overfull hbox/vbox.
%    \begin{macrocode}
%<*latexfilter>
#!/usr/bin/env python3
#coding=utf8

"""

stdin : the original LaTeX log file
stdout : the output filtered log file

"""

from __future__ import print_function
import optparse
import os.path
import re
import sys
import io

def main():
    parser = optparse.OptionParser()
    (options, args) = parser.parse_args()

    display = 0
    in_display = 0
    start_line = ''
    warnerror_re = re.compile(r"^(LaTeX|Package|Class)( (.*))? (Warning:|Error:)")
    fullbox_re = re.compile(r"^(Underfull|Overfull) \\[hv]box")
    accu = ''
    # PDFLaTeX log file is not really in latin-1 (in T1 more exactly)
    # but all bytes are corrects in latin-1, so python won't stop
    # while parsing log.
    # Without specifying this encoding (ie using default utf-8), we
    # can get decode errors (UnicodeDecodeError: 'utf-8' codec can't decode byte...)
    with io.open(sys.stdin.fileno(),'r',encoding='latin-1') as sin:
        for line in sin:
            if display > 0:
                display -= 1
            if line[0:4].lower() in ('info', 'warn') or line[0:5].lower() == 'error':
                display = 0
            line_groups = warnerror_re.match(line)
            if line_groups:
                start_line = line_groups.group(3)
                if not start_line:
                    start_line = ''
                if line_groups.group(2):
                    start_line = "(" + start_line + ")"
                display = 1
                in_display = 1
            elif (start_line != '') and (line[0:len(start_line)] == start_line):
                display = 1
            elif line == "\n":
                in_display = 0
            elif line[0:4] == 'Chap':
                display = 1
            elif fullbox_re.match(line):
                display = 2
            if display:
                print(accu, end="")
                accu = line
            elif in_display:
                print(accu[0:-1], end="")
                accu = line

if __name__ == "__main__":
    main()

%</latexfilter>
%    \end{macrocode}
% \subsection{svgdepth}
%    \begin{macrocode}
%<*svgdepth>
#!/usr/bin/env python3
#coding=utf8

import sys
import xml.parsers.expat


layers = []
for arg in sys.argv:
  layers.append(arg)

parser = xml.parsers.expat.ParserCreate()
class XmlParser(object):
    def __init__(self, layers):
        self.state_stack = [True]
        self.last_state = True
        self.layers = layers
    def XmlDeclHandler(self, version, encoding, standalone):
        sys.stdout.write("<?xml version='%s' encoding='%s'?>\n" % (version, encoding))
    def StartDoctypeDeclHandler(self, doctypeName, systemId, publicId, has_internal_subset):
        if publicId != None: sys.stdout.write("<!DOCTYPE %s PUBLIC \"%s\" \"%s\">\n" %\
            (doctypeName, publicId, systemId))
        else: sys.stdout.write("<!DOCTYPE %s \"%s\">\n" % (doctypeName, systemId))
    def StartElementHandler(self, name, attributes):
        if name.lower() == 'g':
            r = self.last_state and ('id' not in attributes or \
                attributes['id'] in self.layers)
            self.last_state = r
            self.state_stack.append(r)
        if not self.last_state: return
        s = ""
        for k, v in attributes.items(): s += ' %s="%s"' % (k, v)
        sys.stdout.write("<%s%s>" % (name, s))
    def EndElementHandler(self, name):
        r = self.last_state
        if name.lower() == 'g':
            self.state_stack = self.state_stack[0:-1]
            self.last_state = self.state_stack[-1]
        if not r: return
        sys.stdout.write("</%s>" % (name))
    def CharacterDataHandler(self, data):
        if not self.last_state: return
        sys.stdout.write(data)

my_parser = XmlParser(layers)

parser.XmlDeclHandler = my_parser.XmlDeclHandler 
parser.StartDoctypeDeclHandler = my_parser.StartDoctypeDeclHandler 
parser.StartElementHandler = my_parser.StartElementHandler 
parser.EndElementHandler = my_parser.EndElementHandler 
parser.CharacterDataHandler = my_parser.CharacterDataHandler 

for line in sys.stdin:
    parser.Parse(line, False)
parser.Parse('', True)


%</svgdepth>
%    \end{macrocode}
% \Finale
\endinput
